\documentclass[DIN,pagenumber=false,parskip=half,fromalign=left,fromphone=false,fromemail=true,fromurl=false,fromlogo=false,fromrule=false]{scrlttr2}
\usepackage[utf8]{inputenc}
%\usepackage{ngerman}
\usepackage{units}
\usepackage{tabularx}
\usepackage{booktabs}
\usepackage{multirow}
\usepackage{xcolor}
\usepackage{graphicx}
\usepackage[right]{eurosym}
\usepackage{amsmath,amsfonts,amssymb}
\usepackage{braket}
\usepackage{url}
%\RequirePackage{graphicx}

\setkomavar{fromname}{Dr. Elke Faßhauer}
\setkomavar{fromaddress}{Department of Physics and Astronomy\\
                         Aarhus University\\
                         Ny Munkegade 120\\8000 Aarhus\\Denmark}
%\setkomavar{fromphone}{06221/545220}
\setkomavar{fromemail}{elke.fasshauer@gmail.com}
\setkomavar{subject}{Response to Referee Reports}
\setkomavar{signature}{\includegraphics[height=1.5cm]{../pics/u_fasshauer.pdf} \\ Dr. Elke Faßhauer}


\begin{document}
\begin{letter}{To the Editor}
	
	\opening{Dear Editor,}

I am grateful for the feedback from your referees, which I
have taken to heart and
used to improve the manuscript. Please find the full and detailed answers to the
comments below.


\textbf{Reviewer 1}
\begin{enumerate}
 \item \emph{Abstract - the comparison of the derived rule against database
        should be mentioned.}\newline
       Good idea. The last sentence of the abstract now reads:
       We base this rule of thumb on Auger-Meitner decay widths
of Sr$4p^{-1}$ and Ra$6p^{-1}$
obtained by relativistic FanoADC-Stieltjes calculations 
\textcolor{blue}
{and validate it against
Auger-Meitner decay widths from the literature.}
 \item \emph{Page 3: ``One is KNOWN in the literature...'' } \newline
       Corrected.
 \item \emph{Page 5, top: ``... NOBLE gas atoms ...''} \newline
       Corrected.
 \addtocounter{enumi}{1}
 \item 
       \emph{Page 5, ``Moreover, in the single-particle
       picture...'' - this requires a detailed explanation with the exact specification
       of the CK process that is possible for one of the states (e.g. L2), why is
       it forbidden in the single-particle picture, why it starts being energetically
       allowed in the full picture only for high enough Z, etc. Otherwise nothing
       is clear.} \newline
       This remark in its context refers to the Auger-Meitner process from the
       $(n-1)p$ orbitals of the earth alkaline elements only. In order to clarify this
       point, the sentence now reads:
       Moreover, in the single particle picture, a Coster-Kronig decay is possible for neither
\textcolor{red}{of the initial states}
\textcolor{blue}
{of the earth alkaline elements}
and thus the pure Auger-Meitner process, where the initial
states have an equal number of possible decay channels, can be observed.
 \item \emph{Fig. 1 (Sr) - why is there no similar plot for Ra?
             That would be very instructive, for example in view of the discussion
             of the pole strengths.} \newline
             I would like to thank the reviewer for this comment.
             The information this plot illustrates for Sr is that the an Auger
             decay is energetically allowed and that there is only one
             energetically accessible final state. As the reviewer points out,
             it also illustrates relative pole strengths of the initial states, which
             are given in Table I.
             I chose Sr instead of Ra, because
             the channel opening for $s^{-2}$ final states is more likely to be closed
             in the lighter homologues. If the decay channel is open for the light
             homologue, it is also open for the heavier homologue.\\
             With respect to the pole strengths, the required information is the
             manifold of numbers presented in Table I and the new Table III, illustrated
             in Figure 5 and their
             discussions, required for the later analysis.
             The illustration in Figure 5 is more detailed than the ionization
             spectra can be and I have therefore chosen not to include the
             ionization spectra additionally.
 \item \emph{``spin-free" - this is misleading, because spin is taken into account
             (antisymmetry), it is its coupling with the orbital angular momentum that
             is missing.} \newline
             I see, why the reviewer finds the term misleading,
             it is however the
             usual terminology (see Reference 25).
 \item \emph{``ps'' looks like a picosecond. PS?} \newline
             I would like to thank the reviewer for this comment. Another
             common notation of the pole strength is $P$
             (see e.g. J. Chem. Phys. 152, 024125 (2020)). I have chosen this
             notation instead.
 \item \emph{Table I: 3/2, 3/2 and 3/2, 1/2 projections: real part of the energy of
             these degenerate projections is the same, but why imaginary part is
             different - is that an indication of the level of numerical accuracy?
             This must be discussed, as degenerate levels have the same
             (here - complex) energy... } \newline
       I would like to thank the reviewer for this interesting comment. This deviation
       might be used as an indication for numerical accuracy. In this case it would
       result in a numerical error of \unit[2.5]{\%} for strontium
       and \unit[3.0]{\%}
       However, since this observation is not based on physical
       reasoning but only on two numerical data points, I will only indicate that
       this assignment may give an idea about the numerical error and leave the prove
       to future studies to be based on a robust data set.
 \item \emph{Page 9, bottom: The radial density seems to follow the ``general''
             trend in one out of two considered cases - how general is indeed that
             trend then? Or I completely misunderstand what is written... } \newline
             This change of the radial densities is general.
             The difference of the radial expectation value for $Z$ between
             1 and 118 according to Burke
             (Ref. 46) is shown for all orbitals occuring in this paper in the
             following figure:\\
             \includegraphics[width=0.7\textwidth]{../pics/comp_exp_r.pdf}\\
             In order to clarify, I have added the corresponding References 45 and 46
             and added the explanation on page 16.
             
 \item \emph{Page 12 top: ``The analysis of that ... that they are the only other
             contributions ...'' - I do not understand the logical connection.
             The second leading contribution can be at the level of 10\% with
             hundreds of sub-\% other contributions affecting the spread of the
             wavefunction significantly because they involve Rydberg orbitals. }
             \newline
             I would like to thank the reviewer for this comment. She/he
             correctly pointed out that the Rydberg configurations do have a
             significant contribution in the eigenvectors. I have therefore
             included the new Table III, where I present the detailed
             eigenvector analysis. I have furthermore included it in the
             discussion of the decay width results on pages 14 and 16.
             The full analysis does
             not change the main argumentation and results of this manuscript.
 \item \emph{Page 12, bottom: why does not CK process play a role in this rule?
             the explanation is clearly related to (5) above... } \newline
             This is a good question. Unfortunately, the data does
             not support such a simple rule of thumb
             for the CK processes.
 \item \emph{Page 14, top - how well does it hold? What is the mean value and the
             standard deviation of the ration of the two widths over the sample of
             considered elements and shells?} \newline
             This is an interesting question. Connected to it is, which quantity
             is suited for its determination. My rule of thumb includes which
             decay width is larger than the other but not their ratios. I would
             expect that the ratios are the same, because
             many different factors like the element's
             position in the periodic table and, related to it, its electronic 
             structure will influence them. In order to illustrate it,
             I have gathered the means and standard deviations for the three
             different classes in the following table, where
             $\alpha = \frac{\Gamma_{l+1/2}}{\Gamma_{l-1/2}}$.
\begin{center}
\begin{tabular}{lrr}
\toprule
     & $\overline{\alpha} [\unit{eV}]$ & $\sigma [\unit{eV}]$ \\
\midrule
 $L_{2,3}$ & 1.234  & 0.192 \\
 $M_{2,3}$ & 1.653  & 0.808 \\
 $M_{4,5}$ & 1.095  & 0.037 \\
\bottomrule
\end{tabular}
\end{center}
             All means are larger than 1, which can be expected, since the rule
             of thumb holds for all but one of the data points available in the
             literature. For $L_{2,3}$ and $M_{4,5}$, all data points within the standard
             deviation are larger than 1. For $M_{2,3}$ however, the standard deviation
             stretches to values smaller than 1. This is caused by the data point of
             one element (Yb), where the ratio is much larger than in all other cases.\\
             This opens up a the new question, which effect the electronic structure
             plays with respect to the ratios. But since this is not the question raised
             in this manuscript, I will refer to future research. I therefore refer
             to the comparison of the absolute data points, which are included in
             Tables IV-VI and have not included the means and standard deviations
             in the manuscript.
 \item \emph{Page 14, ``However, ...'' - one decay channel LESS than ...} \newline
             Corrected.
 \item \emph{Page 14, ``However....'' - which clear trend (as a function of Z) is expected?
             Connected to (13).} \newline
             I would like to thank the reviewer for this comment. To specifically
             address the expected and actual trends, I have extended the explanation
             on pages 10 as well as 16.
\end{enumerate}



        \closing{Sincerely yours,}
	\end{letter}

\end{document}
