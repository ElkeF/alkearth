\documentclass[DIN,pagenumber=false,parskip=half,fromalign=left,fromphone=false,fromemail=true,fromurl=false,fromlogo=false,fromrule=false]{scrlttr2}
\usepackage[utf8]{inputenc}
%\usepackage{ngerman}
\usepackage{units}
\usepackage{tabularx}
\usepackage{booktabs}
\usepackage{multirow}
\usepackage{xcolor}
\usepackage{graphicx}
\usepackage[right]{eurosym}
\usepackage{amsmath,amsfonts,amssymb}
\usepackage{braket}
\usepackage{url}
%\RequirePackage{graphicx}

\setkomavar{fromname}{Dr. Elke Faßhauer}
\setkomavar{fromaddress}{Department of Physics and Astronomy\\
                         Aarhus University \\
                         Ny Munkegade 120\\8000 Aarhus\\Denmark}
%\setkomavar{fromphone}{06221/545220}
\setkomavar{fromemail}{elke.fasshauer@gmail.com}
\setkomavar{subject}{Submission of my Manuscript}
\setkomavar{signature}{\includegraphics[height=1.5cm]{../pics/u_fasshauer.pdf} \\ Dr. Elke Faßhauer (corresponding author)}


\begin{document}
\begin{letter}{To the Editor}
	
	\opening{Dear Editor,}

I am grateful for the scientific feedback from you and your referees, which I
have used to improve the manuscript. Please find the detailed answers to the
separate points below.

\textbf{Editor}
\begin{itemize}
 \item \emph{Reviewer \#1 is completely correct in his point 1. You could say
             Auger-Meitner spectroscopy/process, but you may not say Meitner
             spectroscopy/process. It is also very disturbing that you do not reference
             the Physics Today article which suggested including Meitner's name.}\\
       It would indeed have been a beneficial to include the reference to the renaming
       proposal (Physics Today, 72, 9. 10 (2019)). It is not the first scientific
       comment on the matter. Especially Duparc's article
       (\url{https://doi.org/10.3139/146.110163}) gives a detailed comparison of what
       the contributions of Lise Meitner, Pierre Auger as well as Ernest Rutherford
       and his student Charles Drumond Ellis were as well as all important references.
       Ellis', Rutherford's and Meitner's main interest was the radioactive $\beta$ decay,
       where the Auger process was observed as a side effect, which was as a first
       correctly described by Meitner in her 1922 and (June) 1923 articles
       (Z. Physik 9 (1922) 131 – 144., Z. Physik 9 (1922) 145 – 152. and
       Z. Physik 17 (1923) 54 – 66.) Pierre Auger focussed on the direct effect
       initiated not by radioactive decay but by ionizing radiation, where he observed
       and described the
       secondary emitted electrons, which he interpreted in the same way as Lise Meitner
       had done. His work was first presented by his
       supervisor on a meeting of the French Academy of Sciences in July 1923
       (C.R.A.S. 177 (1923) 169 – 171.).
       Because Lise Meitner did not attribute her interpretation to be of great enough
       importance to publish it separately Duparc in contrast to the Physics Today Letter
       concludes that the Auger effect was rightly attributed to Pierre Auger alone.\\
       As the authors of the Physics Today article
       I do not agree with Duparc on this attribution. Being the first to correctly
       interprete the discrete part of the electron spectra after radioactive decay,
       as what today is known as Auger process, in my opinion,
       she deserves the same credit as
       Pierre Auger, who focussed on this process after Lise Meitner had interpreted it.\\
       The question whether it should be called Auger-Meitner or Meitner-Auger process
       remains. Reviewer 1 compared the renaming process to the one of the 
       Landau-Zener effect, which was renamed into the
       Landau-Zener-Stückelberg-Majorana effect. Here, the order of the first three
       names is given in chronological order of the submission date of their contribution,
       which is the most accurate data we have available.
       Majorana's contribution was not acknowledged in the same way before 2005
       (\url{https://doi.org/10.3367/UFNr.0175.200505f.0545}). Following the order of the
       first three scientists would suggest to rename the Auger process into the
       Meitner-Auger process, whereas following the example of adding Majoranas due to his
       late acknowleding time would suggest to call it Auger-Meitner process.
       I prefer to acknowledge the scientists in chronological order of their
       contributions, which is why
       I used the term Meitner-Auger process.\\
       {In order to solve this dispute,
       I changed Meitner-Auger into Auger-Meitner throughout the manuscript
       and omit the shortened version.} I have deliberately not marked these changes in
       the Marked Up Manuscript in order to allow you to find those changes that
       lead to a different content more easily.\\
       Moreover, I changed the sentence on page 1, where I
       argue for doing so into:
       In order to give credit, where credit is due
       \textcolor{blue}
       {and following the renaming suggestion of Ref. [3],
       we will refer to it as the Auger-Meitner process.}
       
\end{itemize}

\textbf{Reviewer 1}
\begin{itemize}
 \item \emph{The author aims to formulate a rule of thumb for the relative magnitude of
       the Auger decay widths of p1/2 (L2, M2, N2, ...) and p3/2 (L3, M3, N3, ...)
       atomic levels. Unforunately, a rule of thumb cannot be based on just a couple
       of representative cases, such as considered in detail in the manuscript. One
       either considers tens of cases purely empirically using the tabulated widths or
       derives the rules based on analytical theory, validating them against the full
       database. For L2,3, M2,3, ... Auger decay, the database is available, see
       Atomic Data and Nuclear Data Tables 7, 1 (2001). In that sense, the presented
       work, although clearly warranted and interesting, falls short of reaching the
       proposed goal. Therefore, either the goal needs to be adjusted to a more modest
       one or the work significantly extended.}\\
       I think that the reviewer means Atomic Data and Nuclear Data Tables 77, 1 (2001).
       This work compiles ``recommended'' absolute widths
       being the sum of the widths stemming from
       the Auger-Meitner process,
       the Coster-Kronig process as well as radiative decay. These widths are
       adjusted to yield a total width, which is continuously increasing with $Z$,
       despite the lack of a physical reasoning for doing so and contrasting the
       theoretical data used for the determination of the total widths. This theoretical
       data, however, is very interesting. Unlike in the reference suggested by
       Reviewer 1, it includes the study of numerous, but not all elements.
       Unfortunately, I was only able to find
       the original data of the Auger-Meitner process after initial ionization of the
       $K$, $L$ and $M$ shells (Phys. Rev. A 24, 177 (1981)., Phys. Rev. A 27, 2989 (1983).
       and Phys. Rev. A 21, 449 (1980).). The citations of the Auger-Meitner decay widths
       for primary ionization of the $N$ and $O$ shells do not exist or are given as
       ``private communication'' in 1991 and have not been published by the
       mentioned communication partners between 1979 and 1995. These results would have
       been of particluar interest, because the Coster-Kronig channels are often
       closed for outer-shell vacancies.\\
       I have therefore focussed on the available reliable Auger-Meitner and
       Coster-Kronig decay width data for primary
       $L$ and $M$ shell ionization. They are given in the new tables
       III -- V. The inspection of this data shows that the rule of thumb that I
       formulate in this manuscript holds in all cases but one.
       Details are discussed in the manuscript on page xyz .\\
       I have included a description of the (Super-)Coster-Kronig process on page xyz,
       clarified that the rule of thumb is to hold for Auger-Meitner and not
       total decay widths on page 1.
\end{itemize}

\begin{enumerate}
 \item \emph{Rebranding the Auger effect requires an authority of the same calibre
       or higher than Auger, Meitner or Wentzel. One good reason to refrain from it,
       even if such an authority was available, is the basic respect for the contributions
       of many scientists who have been working on the fundamentals and applications
       of Auger physics and collectively produced results no weaker than the three
       founders of the field had originally. Even if I had thought that the rebranding
       was possible, it would have had to be an additive one, as we have witnessed
       for example in the case of Landau-Zener effect: Landau-Zener-Stuckelberg-Majorana.}\\
       See above under the editor's comment.
 \item \emph{[4] is not a time resolved experiment, but rather a quantum mechanical
       theory of the experiment by Drescher et al. Nature (2002). It is indeed important
       to cite Ref. [4], but the original experimental reference cannot be missing.}
 \item \emph{[10-13] there are plenty of ADC references in literature and if one
       chooses to cite a few of them, the selection should reflect the fact that the
       ADC is a joint invention of Schirmer and Cederbaum. More importantly ADC is
       by far not the only approach to calculate the initial and final state energies
       as may be understood from that sentence. One could think of EOM-CC, CI, Delta SCF
       or MCSCF... The basic reference for Fano-ADC is missing
       [J. Chem. Phys. 123, 204107 (2005)]. Compare Eqs.(1-6) of the present manuscript
       to Fano-ADC equations in the 2005 paper.}
 \item \emph{Section III - the author needs to expand on the moments with numerical
       instabilities.}
 \item \emph{Why is the radial density of p1/2 orbital closer to nucleus than that of
       the p3/2 orbital? How does the analytical solution [38] show this? A simple
       physical explanation is missing.}
 \item \emph{Fano-ADC is not limited to any particular order. The implementation
       used by the author and others before is indeed limited, but this is not an
       essential limitation of the approach.}
 \item \emph{Surely the rule of thumb should include the dependence on Z. What does
       the table of L2/L3 level widths available in the literature (see above) say
       about the L2/3 levels of the low Z atoms?}
 \item \emph{How can we disregard the configuration mixing effect in the initial
       state on the basis of just two atomic calculations alone?}
\end{enumerate}

\textbf{Referee 2}
\begin{enumerate}
 \item o
\end{enumerate}



        \closing{Sincerely yours,}
	\end{letter}

\end{document}
