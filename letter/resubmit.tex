\documentclass[DIN,pagenumber=false,parskip=half,fromalign=left,fromphone=false,fromemail=true,fromurl=false,fromlogo=false,fromrule=false]{scrlttr2}
\usepackage[utf8]{inputenc}
%\usepackage{ngerman}
\usepackage{units}
\usepackage{tabularx}
\usepackage{booktabs}
\usepackage{multirow}
\usepackage{xcolor}
\usepackage{graphicx}
\usepackage[right]{eurosym}
\usepackage{amsmath,amsfonts,amssymb}
\usepackage{braket}
\usepackage{url}
%\RequirePackage{graphicx}

\setkomavar{fromname}{Dr. Elke Faßhauer}
\setkomavar{fromaddress}{Department of Physics and Astronomy\\
                         Aarhus University\\
                         Ny Munkegade 120\\8000 Aarhus\\Denmark}
%\setkomavar{fromphone}{06221/545220}
\setkomavar{fromemail}{elke.fasshauer@gmail.com}
\setkomavar{subject}{Response to Referee Reports}
\setkomavar{signature}{\includegraphics[height=1.5cm]{../pics/u_fasshauer.pdf} \\ Dr. Elke Faßhauer}


\begin{document}
\begin{letter}{To the Editor}
	
	\opening{Dear Editor,}

I am grateful for the scientific feedback from you and your referees, which I
have taken to heart and
used to improve the manuscript. Please find the fulle and detailed answers to the
comments below.

\textbf{Editor}
\begin{itemize}
 \item \emph{You could say
             Auger-Meitner spectroscopy/process, but you may not say Meitner
             spectroscopy/process. It is also very disturbing that you do not reference
             the Physics Today article which suggested including Meitner's name.}\\
%       It would indeed have been a beneficial to include the reference to the renaming
%       proposal (Physics Today, 72, 9. 10 (2019)). It is not the first scientific
%       comment on the matter. Especially Duparc's article
%       (\url{https://doi.org/10.3139/146.110163}) gives a detailed comparison of what
%       the contributions of Lise Meitner, Pierre Auger as well as Ernest Rutherford
%       and his student Charles Drumond Ellis were as well as all important references.
%       Ellis', Rutherford's and Meitner's main interest was the radioactive $\beta$ decay,
%       where the Auger process was observed as a side effect, which was as a first
%       correctly described by Meitner in her 1922 and (June) 1923 articles
%       (Z. Physik 9 (1922) 131 – 144., Z. Physik 9 (1922) 145 – 152. and
%       Z. Physik 17 (1923) 54 – 66.) Pierre Auger focussed on the direct effect
%       initiated not by radioactive decay but by ionizing radiation, where he observed
%       and described the
%       secondary emitted electrons, which he interpreted in the same way as Lise Meitner
%       had done. His work was first presented by his
%       supervisor on a meeting of the French Academy of Sciences in July 1923
%       (C.R.A.S. 177 (1923) 169 – 171.).
%       Because Lise Meitner did not attribute her interpretation to be of great enough
%       importance to publish it separately Duparc in contrast to the Physics Today Letter
%       concludes that the Auger effect was rightly attributed to Pierre Auger alone.\\
%       As the authors of the Physics Today article
%       I do not agree with Duparc on this attribution. Being the first to correctly
%       interprete the discrete part of the electron spectra after radioactive decay,
%       as what today is known as Auger process, in my opinion,
%       she deserves the same credit as
%       Pierre Auger, who focussed on this process after Lise Meitner had interpreted it.\\
%       The question whether it should be called Auger-Meitner or Meitner-Auger process
%       remains. Reviewer 1 compared the renaming process to the one of the 
%       Landau-Zener effect, which was renamed into the
%       Landau-Zener-Stückelberg-Majorana effect. Here, the order of the first three
%       names is given in chronological order of the submission date of their contribution,
%       which is the most accurate data we have available.
%       Majorana's contribution was not acknowledged in the same way before 2005
%       (\url{https://doi.org/10.3367/UFNr.0175.200505f.0545}). Following the order of the
%       first three scientists would suggest to rename the Auger process into the
%       Meitner-Auger process, whereas following the example of adding Majoranas due to his
%       late acknowleding time would suggest to call it Auger-Meitner process.
       I fully agree and have therefore added the reference to the manuscript.
       Even though I personally prefer to acknowledge the involved scientists in
       chronological order of their contributions
       in the naming of this decay process, I have changed Meitner-Auger into
       Auger-Meitner throughout the manuscript.
       I prefer to acknowledge the scientists in chronological order of their
       contributions, which is why
       I used the term Meitner-Auger process.\\
       I have deliberately not marked these changes in
       the Marked Up Manuscript in order to allow you to find those changes that
       lead to a different content more easily.
\end{itemize}

\textbf{Reviewer 1}
\begin{enumerate}
 \item \emph{The author aims to formulate a rule of thumb for the relative magnitude of
       the Auger decay widths of p1/2 (L2, M2, N2, ...) and p3/2 (L3, M3, N3, ...)
       atomic levels. Unforunately, a rule of thumb cannot be based on just a couple
       of representative cases, such as considered in detail in the manuscript. One
       either considers tens of cases purely empirically using the tabulated widths or
       derives the rules based on analytical theory, validating them against the full
       database. For L2,3, M2,3, ... Auger decay, the database is available, see
       Atomic Data and Nuclear Data Tables 7, 1 (2001). In that sense, the presented
       work, although clearly warranted and interesting, falls short of reaching the
       proposed goal. Therefore, either the goal needs to be adjusted to a more modest
       one or the work significantly extended.}\\
       I would like to thank the reviewer for this constructive critizism.
       Atomic Data and Nuclear Data Tables 77, 1 (2001)
       compiles ``recommended'' absolute widths
       being the sum of the widths stemming from
       the Auger-Meitner process,
       the Coster-Kronig process as well as radiative decay. These widths are
       adjusted to yield a total width, which is continuously increasing with $Z$,
       despite the lack of a physical reasoning for doing so and contrasting the
       theoretical data used for the determination of the total widths. This theoretical
       data, however, is very interesting. Unlike in the above refererence suggested by
       Reviewer 1, it includes the study of numerous elements.
       Unfortunately, I was only able to find
       the original data of the Auger-Meitner process after initial ionization of the
       $K$, $L$ and $M$ shells (Phys. Rev. A 24, 177 (1981)., Phys. Rev. A 27, 2989 (1983).
       and Phys. Rev. A 21, 449 (1980).). The citations of the Auger-Meitner decay widths
       for primary ionization of the $N$ and $O$ shells do not exist or are given as
       ``private communication'' in 1991 and have not been published by the
       mentioned communication partners between 1979 and 1995. These results would have
       been of particluar interest because the Coster-Kronig channels are often
       closed for outer-shell vacancies or contribute to both initial states.\\
       I have therefore focussed on the available reliable Auger-Meitner and
       Coster-Kronig decay width data for primary
       $L$ and $M$ shell ionization. They are given in the tables
       III--V of the revised manuscript.
       The inspection of this data shows that the rule of thumb that I
       formulated based on my calculations and their analysis holds in all cases but one.
       I have thereby both significantly extended the manuscript to fit the proposed
       goal.
       Details are discussed in the manuscript on pages 12--14 .\\
       I have included a description of the (super-)Coster-Kronig process and
       clarified that I differentiate between Auger-Meitner processes on pages 4--5.
 \item \emph{Rebranding the Auger effect requires an authority of the same calibre
       or higher than Auger, Meitner or Wentzel. One good reason to refrain from it,
       even if such an authority was available, is the basic respect for the contributions
       of many scientists who have been working on the fundamentals and applications
       of Auger physics and collectively produced results no weaker than the three
       founders of the field had originally. Even if I had thought that the rebranding
       was possible, it would have had to be an additive one, as we have witnessed
       for example in the case of Landau-Zener effect: Landau-Zener-Stuckelberg-Majorana.}\\
       See response to the Editor above.
 \item \emph{[4] is not a time resolved experiment, but rather a quantum mechanical
       theory of the experiment by Drescher et al. Nature (2002). It is indeed important
       to cite Ref. [4], but the original experimental reference cannot be missing.}\\
       This is a good point. The reference was added on page 2.
 \item \emph{[10--13] there are plenty of ADC references in literature and if one
       chooses to cite a few of them, the selection should reflect the fact that the
       ADC is a joint invention of Schirmer and Cederbaum. More importantly ADC is
       by far not the only approach to calculate the initial and final state energies
       as may be understood from that sentence. One could think of EOM-CC, CI, Delta SCF
       or MCSCF... The basic reference for Fano-ADC is missing
       [J. Chem. Phys. 123, 204107 (2005)]. Compare Eqs.(1--6) of the present manuscript
       to Fano-ADC equations in the 2005 paper.}\\
       I would like to thank the reviewer for pointing this out.
       To give an overview of all ADC references, I have added the book by J. Schirmer
       as well as the standard 1983 reference and clarified the sentence.
       The original FanoADC reference was initially not given, as the context
       in the manuscript are relativistic approaches.
       But I do agree that it deserves the credit and
       have added the reference
       on page 3.
 \item \emph{Section III - the author needs to expand on the moments with numerical
       instabilities.}\\
       I followed the same approach as described in detail in section II C of Ref. [23].
       I therefore added the reference in section III on page 7.
 \item \emph{Why is the radial density of p1/2 orbital closer to nucleus than that of
       the p3/2 orbital? How does the analytical solution [38] show this? A simple
       physical explanation is missing.}\\
       I would like to thank the reviewer for this comment.
       I checked my analytical expressions and found a mistake. Indeed, for the
       one-electron ansatz $\rangle r \langle$ is larger for the $l-1/2$ case than for the $l+1/2$
       case. This result is inverted by taking other electrons into account even by
       SCF simulations.
       I have therefore removed the comment on page 9 from the manuscript.
       This finding also illustrates
       why a purely analytical foundation for the rule of thumb is not possible.
 \item \emph{Fano-ADC is not limited to any particular order. The implementation
       used by the author and others before is indeed limited, but this is not an
       essential limitation of the approach.}\\
       The reviewer is correct and I clarified this part in the text on page 11.
 \item \emph{Surely the rule of thumb should include the dependence on Z. What does
       the table of L2/L3 level widths available in the literature (see above) say
       about the L2/3 levels of the low Z atoms?}\\
       As can be seen from the data in the new Tables III--V, there is no clear
       dependence on $Z$. I have included a comment about it on page 14 of the
       revised manuscript.
 \item \emph{How can we disregard the configuration mixing effect in the initial
       state on the basis of just two atomic calculations alone?}\\
       The relevant question is not whether configuration mixing occurs but whether this
       results in an equal number of additional decay channels or not. For core levels,
       orbital mixing is negligible. For the one or two outermost shells,
       the energies of the relevant configurations are close to the energy of the main
       configuration. The kinetic energy of the atomic
       Auger-Meitner electrons is usually much higher than the energy splitting of different
       configurations. The additional decay
       channels available due to the mixing are therefore energetically accessible
       to both spin-orbit split initial states.
       As I have shown for the case of Sr and Ra based on the
       radial orbital densities, the increased overlap of the wavefunction of the
       hole with the final state increases in the same way for the excited
       configurations. Independent of the composition of these configurations, the
       radial expectation values of the involved orbitals will always be larger than
       those of the initial inner vacancies and therefore support my rule of thumb.
       The assessment of the mixing effect is based on this reasoning.
\end{enumerate}


\textbf{Reviewer 2}

\begin{enumerate}
 \item \emph{My most significant concern is whether the role of thumb which apparently
             the main goal of this work and is reported also in the conclusion as
             ``Two ionized initial states that stem
             from the same non-relativistic configuration and are split by
             spin-orbit coupling will have
             different decay widths, where the decay width of the $l - 1$ initial
             state will be significantly
             lower than the decay width of the $l + 1$ initial state.''
             can actually be drawn from two atomic calculations.}\\
             I would like to thank the reviewer for this comment.
             I have added the discussion of Auger-Meitner decay rates of multiple
             other elements.
             See response to the Editor above.
 \item \emph{Also, its potential generalization to the ICD or ETMD processes
             (see page 12), certainly of great interest,
             should be proved.
             In my opinion the work needs to be revised and washed out from some
             generalizations from the author
             which seems a bit of a stretch and not supported by data.
             These generalizations are not necessary and do not improve the quality
             of the work.}\\
             I agree that a validation of the extension to the ICD and ETMD processes
             is necessary. The reasoning would be the same as for the Auger process
             and hence, this part was meant as an outlook. I have clarified my intention
             on page 14.
 \item \emph{affiliation b) ?}\\
       Point taken, removed.
 \item \emph{I appreciate the aim of the author to give credit to Lise Meitner for
             her discovery of the Auger process. I know that there is a proposal
             to rename the Auger effect in
             Auger-Meitner (Physics Today, 72, 9, 10 (2019)) (and not as Meitner-Auger
             as many times used in the work).
             I found a bit confusing the exclusive use of the name of Meitner.
             I had never heard the
             ``Meitner spectroscopy'', in my view may be misleading.}\\
      See above under the Editor's comment.
\item \emph{``spin-free Hamiltonian'' please give a reference.}\\
      The reference was added and acknowledgements are given to Dyall on page 4.
\item \emph{page 4 ``varify'' maybe should read ``verify''}\\
      Done.
\end{enumerate}

        \closing{Sincerely yours,}
	\end{letter}

\end{document}
