\documentclass[DIN,pagenumber=false,parskip=half,fromalign=left,fromphone=true,fromemail=true,fromurl=false,fromlogo=false,fromrule=false]{scrlttr2}
\usepackage[utf8]{inputenc}
\usepackage{ngerman}
\usepackage{units}
\usepackage{tabularx}
\usepackage{booktabs}
\usepackage{multirow}
\usepackage{xcolor}
\usepackage{graphicx}
\usepackage[right]{eurosym}
\usepackage{amsmath,amsfonts,amssymb}
\usepackage{braket}
%\RequirePackage{graphicx}

\setkomavar{fromname}{Dr. Elke Faßhauer}
\setkomavar{fromaddress}{CTCC\\University of Troms\o --- The Artic University of Norway\\9037 Tromsø\\Norway}
%\setkomavar{fromphone}{06221/545220}
\setkomavar{fromemail}{elke.fasshauer@uit.no}
\setkomavar{subject}{Revision of Our Article No. A14.12.0322}
\setkomavar{signature}{Dr. Elke Faßhauer (corresponding author)}


\begin{document}
\begin{letter}{To the Editor}
	
	\opening{Dear Editor,}


Our paper has been reviewed by two of your referees,
who recommended it for the publication in the
Journal of Chemical Physics subject to minor revisions.
We are thankful for the comments made by the referees and
respond to them below. We took the opportunity to improve formulations
and correct some typos.

Referee \#1 raised the following points with the manuscript:

\begin{enumerate}
 \item \emph{Theory. Part B. Page 10: The criteria used to choose the
             \emph{possible configurations} should be explicitly specified.}
       We added some more information and made the choice of configurations more
       clear. The paragraph now reads:\\
       {\color{blue}{Not all $2h1p$ states describe energetically allowed final
       states.
       The energy conservation $E_{in} = E_{A^{2+}} + E_{e^-_{sec}}$ has
       to be satisfied, which is only possible for $E_{A^{2+}} < E_{in}$, where
       $E_{in}$ denotes the initial state energy, $E_{A^{2+}}$ the energy of the
       doubly ionized part of the final state and $E_{e^-_{sec}}$ denotes the
       kinetic energy of the secondary electron emitted in the process.
       If the energy of
       the doubly charged system is higher than the energy of the initial state, the
       process is energetically forbidden. Hence, all $2h1p$ states characterized by
       a $2h$ contribution with an energy higher than the initial state energy will
       not contribute to the final states.Instead, these $2h1p$ states can be
       used to improve the description of
       the initial state.
       The partitioning into initial and final state configurations can be
       achieved based on Hartree-Fock orbital occupations by manual choice of $2h$
       configurations
       or in an energy-based approach as described by Averbukh [44].
       In this work, we employ the partitioning by population and manually include
       those configurations which correspond to open channels.}}
 \item \emph{Computational details. Page 13: \emph{The resulting moments were
             checked for the numerical instabilities}. Some details of the
             numerical instability analysis would be helpful.}\\
       We added further information about it at the end of the theory part
       the last paragraphs now read:\\
       Since the integral was evaluated and not the density function as such
       the distribution function obtained from the
       discrete pseudo-spectrum is normalized correctly (see [53]).
       This distribution function is then numerically differentiated via
       
       \begin{equation}                                                  
         f^{(n)} (\varepsilon) =                                              
         \begin{cases}                                                   
           \frac 12 \frac{f_1}{\varepsilon_1}    & \varepsilon < \varepsilon_1\\        
           \frac 12 \frac{f_{i+1} + f_i}{\varepsilon_{i+1} - \varepsilon_i}        
                                          & \varepsilon_i < \varepsilon < \varepsilon_{i+1}\\
           0                              & \varepsilon_n < \varepsilon
         \end{cases}                                                     
       \end{equation}
       
       to give  $r-1$ non-zero points of the desired
       density function $f(\varepsilon)$, which are
       subsequently interpolated.
       In the routine of Averbukh, a
       monotonicity-preserving piecewise cubic Hermite spline interpolation
       is used for this purpose. Afterwards, the interpolated density function is evaluated
       for the energy of interest, which is the resonance energy $E_r$ in case of the
       autoionization processes to give the decay width $\Gamma$.
       {\color{blue}{Finally, convergence of the decay widths with increasing orders of 
       moments is investigated.}}
       
       Unfortunately the procedure to obtain the optimal abscissae and weights
       involves the subtraction of two large numbers, which
       leads to numerical instabilities in high order moments. Therefore, they have to
       be checked carefully and only trustworthy moments should be
       used for the evaluation of the decay widths.
       {\color{blue}{This can be achieved by inspection of abscissae and weights
       of each order.
       Since the decay width $\Gamma(E)$ is a smooth function
       unphysical oscillations in the curve constructed
       from the abscissae and weights of one order of moments indicate numerical
       instabilities in this particular order.
       If this behaviour is observed, the abscissae and weights obtained from this
       order of moments and all higher orders are discarded.
       Finally, the abscissae
       and weights from the remaining, consecutive orders of moments enter
       the interpolation scheme.}}
 \item \emph{The manuscript contains multiple typos, and should be proof-read
             one more time.}
   The following typos were corrected:
   \begin{itemize}
    \item ... or hydrogen {\color{blue}{fluoride}} ...
    \item ... than spherical symmetry. This reduces the {\color{blue}{applicability}}
          of the method to ...
    \item ... the $(n-1)d^{-1}$ ionization of the noble gases krypton,
          {\color{blue}{xenon}} and radon, ...
    \item ... processes first {\color{blue}{proposed}} by Hazi [48],
          be solved by using the ...
    \item Only those moments, without {\color{blue}{numerical}} instabilities
          entered the interpolation ...
    \item ... obtained with the {\color{blue}{relativistic}} FanoADC in this work,
          the {\color{blue}{MMCDF}} ...
    \item ... in the {\color{blue}{initial}} state description as well.
    \item The {\color{blue}{non-relativistically}} obtained results are
          lower than those ...
   \end{itemize}
\end{enumerate}

Referee \#2 raised several different points within one comment. In this letter
we split these comments and discuss the points seperately in the following. 

\begin{enumerate}
 \item \emph{Key parts of the theoretical and computational development given
             in Sections II B and C reduce the many Auger final-channels present
             (Figure 4) to a single effective channel form, apparently following
             the ADM methods reported in reference [44], continuum functions for
             which are constructed employing a highly primitive version of
             Stieltjes methodologies [54]. The description of the ADM channel
             reduction approach is not particularly clear and needs amplification,
             particularly since the importance of final-channel coupling,
             relaxation, and other effects in atomic Auger decay have been known
             since at least the 1970s [eg., J. Phys. B: At. Mol. Phys. 11, 1575
             (1978)], and in some instances well before then. The approach
             employed here [44] to collapse these effects into a single effective
             channel and total decay width is apparently ad hoc in nature and
             performed in the context of the ADM calculations, rather than within
             the multi-channel Feshbach-Fano methodology cited, rendering its
             validity uncertain. Since a complete quantum-mechanical treatment
             of multi-channel phenomena employing the modern Stieltjes formalism
             and Fano-Cooper interaction-prepared states [Phys. Rev. 137, A1364
             (1965), Rev. Mod. Phys. 40, 441 (1968)] has long been available in
             the literature, and correct expressions for total decay widths
             reported [Com. Phys. Comm. 53, 123 (1989), see also, Chem. Phys.
             Lett. 151, 417 (1988), Physica Scripta T31, 179 (1990), J. Chem.
             Phys. 95, 3107 (1991] comparisons with this general approach
             identifying the difference or similarities with the ad hoc ADM
             procedure will considerably strengthened the manuscript and are
             clearly in order.}\\
         The principal goal of Fano-ADC methods is the evaluation of total decay
         width, which, while clearly not providing complete description of the
         process, still represents valuable information directly related to the
         lifetime of the metastable state. The apparent reduction of the decay
         rate formula to single effective channel form (Eq.20 in [44]) is an
         incorrect interpretation. As discussed in [44], for the total decay
         width the present approach is fully equivalent to the multichannel
         theory of JPB 11,
         1575 (1978). Due to the use of correlated eigenfunctions of the
         projected ADC Hamiltonian for initial (discrete) and final states,
         discussed effects as channel coupling and relaxation are taken into
         account at the level that corresponds to the order of ADC scheme
         employed. The simple form of the working formula (20) in reference [44]
         for the
         spectral moments of decay widths stems from the fact
         that within the finite interaction region relevant for the evaluation of
         $\braket{\Phi|V|\chi_{\beta,\varepsilon}}$ couplings,
         the exact expansion of unity
         
         \begin{equation*}
          \sum_\beta \int d\varepsilon \ket{\chi_{\beta,\varepsilon}}
                                       \bra{\chi_{\beta,\varepsilon}}
         \end{equation*}
         
         can be to a good approximation replaced in terms of ADC eigenstates
         
         \begin{equation*}
          \sum_q \ket{\chi^{2h1p}_q} \bra{\chi^{2h1p}_q}
         \end{equation*}
         
         as long as a good enough basis of intermediate states is used. It is this
         key observation that allows us to evaluate the spectral moments of the
         total decay width using [44].(20). Advanced Stieltjes imaging technique
         of Refs. [50-52,57], which is briefly summarized in the present paper,
         is then used to reproduce the decay width. The straightforward approach
         of Hazi [JPB 11, L259 (1978)], probably referred to by the referee as
         \emph{highly primitive version of Stieltjes methodologies}, is of course not
         used in present Fano-ADC-Stieltjes methods.
         
         Other references mentioned by the referee are not directly relevant to
         the present paper but rather target specifically the problem of
         photoionization. While there is a large overlap in the use of
         Feshbach-Fano theory of resonances and Stieltjes techniques allowing the
         use of $\mathcal{L}^2$ basis for the representation of continuum,
         the advances made
         in these papers cannot be employed straightforwardly to the problem of
         autoionization. Of course, the optimal definition of the discrete state
         representing the resonance is an important problem strongly affecting
         the convergence of the Stieltjes imaging. However, the strategies
         proposed in the considered references cannot be used for the
         representation of cationic autoionizing states.
         Similarly, as stated in JCP 95,
         3107 (1991), the construction of Stieltjes-Chebyshev continuum states
         is only needed for complete specification of the photoinonization cross
         section including the non-resonant (background) contribution,
         but not for evaluation of the resonance width. The prospects of
         similar construction to provide more accurate partial decay width than
         the non-rigorous approach described in [44] is appealing but beyond the
         scope of present paper.\\
         In order to make these points more clear
         in the manuscript, we changed two sentences in the first
         paragraph of II.B to:
     \begin{itemize}
       \item In the FanoADC, the discrete ADC Hamiltonian
             is used for the construction of the pseudo-spectrum
             {\color{blue}{for the evaluation of the total}} decay width $\Gamma$.
       \item Therefore, the {\color{blue}{basis functions for the description of
             the entire final state subspace}} are chosen from
             the $2h1p$ class of configurations, where the $2h$ part is assumed to
             describe the doubly ionized final state and the
             particle represents the continuum electron. This approach can be
             justified by rewriting the expression in Eq. (1) explicitly.
     \end{itemize}
         We would like to stress that the aim of this work is not to improve the
         concept of the
         FanoADC-Stieltjes method, which has been shown to provide accurate
         results for light elements and small molecules (see
         e.g. JCP 135, 134314 (2011)), but to present its implementation in the
         program package Dirac, which allows for the calculation of decay
         widths including relativistic effects on the four-component level of
         theory. In order to achieve clarity, the abstract was rewritten (see
         below for changes). Additionally an explanation of the no-pair
         approximation was added to the theory part:\\
         These equations can also be used in the relativistic case based on
         Dirac-Hartree-Fock orbital energies as well as integrals
         {\color{blue}{
         [33,34]  and using the
         no-pair approximation (see e.g. [27]).
         This
         approximation ensures that pair creation processes are excluded from
         the calculation by allowing annihilation and creation operators
         $c_p$ and $c_q^\dagger$ in the
         spectral representation
         \begin{equation}
          G^-_{pq}(\omega) = \sum\limits_{n\in \{N-1 \}}
                             \frac{\braket{\Psi_0^N| c_q^\dagger | \Psi_n^{N-1}}
                                   \braket{\Psi_n^{N-1}| c_p |\Psi_0^{N-1}}}
                             {\omega + E_n^{N-1} - E_0^N -i\eta}
         \end{equation}
         
         of Eq. (7)
         to operate on the space of positive energy solutions only. Since energies
         high enough to overcome the gap of $2mc^2$ are hardly achieved in chemistry,
         this approximation is reasonable.

         }}
 \item \emph{The Abstract is important and could be rewritten to achieve greater
             clarity. Its not clear, for example, if \emph{This
             non-relativistically
             established FanoADC-Stieltjes procedure was implemented into the
             program package Dirac} implies a non-relativistic code is being
             added to Dirac in the present work, whereas, in fact, a relativistic
             update is being added. The sentence, \emph{In contrast to
             existing atomic
             codes, it can be used for the calculation of lifetimes of systems
             with less than spherical symmetry} probably means something like
             \emph{the new code can be employed for atoms, molecules, clusters and
             other general spatial symmetries} presumably also including
             relativistic effect? The Abstract could be a bit longer and include
             other high points of the study, perhaps taken from the more clearly
             written Introduction, but this is only a suggestion. I do have
             some misgivings about the text more generally and the paucity of
             references to earlier work on Ke, Xe, Rn - surely other calculations
             and measurements of Auger widths are available other those reported
             in Table I. Additionally, Eq. (1) in the context of the Auger effect
             is common attributed to G. Wentzel [Z. Physik, 43, 524 (1927)],
             whereas the Fano and Feshbach formalisms are more generic in purpose.
             The authors may want to walk through the manuscript and remove
             misstatements; eg., \emph{matrix elements are only non-zero within an
             interaction region of finite size} on page 6 doesn't really mean
             anything. Finally, there may be value in citing the on-line
             dissertation of Elke Fasshauer, which contains lots of useful
             relevant information.}\\
  \begin{enumerate}
   \item \textbf{Abstract}
         We changed the second and third paragraph of the abstract to:\\
         We present a {\color{blue}{realization of the non-relativistically
         established FanoADC-Stieltjes method for the description of autoionization
         decay widths including relativistic effects.
         This procedure, being based on the
         Algebraic Diagrammatic Construction (ADC),
         was adapted to the relativistic framework and implemented
         into the relativistic quantum chemistry program package Dirac.
         It is, in contrast to other existing relativistic atomic codes,
         not limited to the
         description of autoionization lifetimes in spherically symmetric systems,
         but is instead also applicable to molecules and clusters.\\
         We employ}} this method to the Auger processes following the Kr3d$^{-1}$,
         Xe4d$^{-1}$ and Rn5d$^{-1}$ ionization. Based on the results we
         show a pronounced
         influence of mainly scalar-relativistic effects
         on the decay widths of autoionization processes.
   \item \textbf{Further calculations and measurements of decay widths}
         We carefully checked the work on earlier measurements and calculations
         of decay widths and we indeed found one further publication of calculated
         values for krypton. However, the references for xenon are up-to-date and
         no numbers were to be found for radon. We therefore added these numbers
         to Table I, changed the caption of the table and
         added a discussion of these results. It now reads:\\
         {\color{blue}{
         Total Auger decay widths of the Kr, Xe and Rn Rg$(n-1)$d$_{5/2}^{-1}$
                   and Rg$(n-1)$d$_{3/2}^{-1}$
                   and the non-relativistic
                   Rg$(n-1)$d$^{-1}$ initial states
                   compared to theoretical values for krypton [62],
                   xenon [63] and  experimental values for xenon
                   [64].
                   All widths are given in \unit{meV}.\\

          \begin{tabular}{llccr}                                            
           \toprule                                                         
            $\Gamma$            & initial state   &    exp.    & theo.& this work\\       
           \midrule                                                         
            \multirow{4}{*}{Kr} & 3d$_{5/2}$      &     --     &  51  & 63\\
                                & 3d$_{3/2}$      &     --     &  49  & 56\\
                                & 3d$_{spinfree}$ &     --     &  --  & 62\\
                                & 3d$_{nrel}$     &     --     &  --  & 49\\
           \midrule                                                         
            \multirow{4}{*}{Xe} & 4d$_{5/2}$      & 110 -- 130 & 160  & 162\\
                                & 4d$_{3/2}$      & 105 -- 116 & 143  & 132\\
                                & 4d$_{spinfree}$ &     --     &  --  & 168\\
                                & 4d$_{nrel}$     &     --     &  --  & 90\\
           \midrule                                                         
            \multirow{4}{*}{Rn} & 5d$_{5/2}$      &     --     &  --  & 624\\
                                & 5d$_{3/2}$      &     --     &  --  & 547\\
                                & 5d$_{spinfree}$ &     --     &  --  & 686\\
                                & 5d$_{nrel}$     &     --     &  --  & 161\\
           \bottomrule                                                      
          \end{tabular}                                                     

         For krypton one other set of calculated decay widths is available in
         the literature [62]. These calculations are based on a
         many-body perturbation approach including all orders of perturbation.
         They state that these results are approximately 75\% of the values
         for calculations including   
         perturbations up to second order only. Therefore the second order numbers
         would agree well with our values obtained with the FanoADC-Stieltjes
         method shown       
         in Table I. However, all integrations in this procedure rely
         on analytic evaluation using angular momentum algebra and are hence not
         applicable to non-spherical systems.}}
   \item \textbf{Equation (1)}
         The expression given in Eq. (1) is indeed the same as in the formulation
         of Wentzel. However, the result for autoionization processes
         in the more sophisticated ansatz of Feshbach and Fano is identical
         to the one of Wentzel. Since we in the method apply the partitioning
         into initial and final state subspaces inherent to Feshbach and Fano,
         we cite Feshbach and Fano here. However, we added Wentzel.
         The first sentence of the theory part now reads:\\
         Following {\color{blue}{Wentzel [47] and later}} Feshbach [36,38]
         and Fano [37] ...
   \item \textbf{Non-zero matrix elements}:
         This sentence does have meaning because it once again illustrates, why
         it is possible to gain information about continuum properties with
         localized basis functions. To make this point more clear, we changed
         the paragraph to:\\
         It relies on the observation that the moments
         {\color{blue}{ of the projected final
         state Hamiltonian $H_f$}}
         
         \begin{equation}
          \mu_k = \braket{ \Phi | \hat{V} H^k_f \hat{V} | \Phi }          
         \end{equation}                                                   
                                                                          
         calculated from the determined
         discrete pseudo-spectrum are good approximations to the moments determined
         from the real continuum states.                                  
         This can be shown by {\color{blue}{inserting}} the resolution of identity for 
         the continuum states

         {\color{blue}{\begin{equation*}
          \mu_k = \sum_i \varepsilon_i^k
                  \left| \braket{ \Phi | \hat{V} | \chi_{i,\varepsilon} } \right| ^2
                + \int\limits_{E_{thr}}^{\infty} \varepsilon^k
                  \left| \braket{\Phi|\hat{V}|\chi_{\varepsilon}} \right|^2 \mathrm{d}\varepsilon .
         \end{equation*}
         
         Since the non-zero contribution to the coupling matrix elements in the
         Feshbach-Fano approach stem only
         from an interaction region of finite size, where the $\mathcal{L}^2$ final
         state functions are nonvanishing, we may replace the expansion
         $\sum\limits_i \ket{\chi_{i,\varepsilon}} \bra{\chi_{i,\varepsilon}}
          + \int \mathrm{d}\varepsilon \ket{\chi_\varepsilon} \bra{\chi_\varepsilon}$
         by its $\mathcal{L}^2$ approximation
         $\sum\limits_j \ket{\tilde{\chi}_{\tilde{E}_j}} \bra{\tilde{\chi}_{\tilde{E}_j}}$
         (see [53])
         
         \begin{equation*}
          \label{eq:moment_discrete}
          \mu_k = \sum\limits_j \tilde{E}_j ^k
                  \left| \braket{\Phi|\hat{V}|\tilde{\chi}_{\tilde{E}_j}}  \right|^2 .
         \end{equation*}
         
         Then the decay width can be determined through a series of
         consecutive approximations to the moments of increasing order $k$.}}
   \item \textbf{Citation of thesis}:
         We added the following sentence to the introduction to the theory part:\\
         To achieve this kind of description, we {\color{blue}{choose}} the non-relativistic
         FanoADC-Stieltjes approach, described                            
         in the next sections. Here, the ADC is used for the description of the          
         initial and final states and the resulting discrete pseudo-spectrum
         is then subject to a Stieltjes imaging procedure.
         {\color{blue}{An exhaustive description of the method can be found in [54].}}
  \end{enumerate}
\end{enumerate}


We took the opportunity to improve the manuscript in the following way:

\begin{itemize}
 \item Afterwards, the actual Auger process of the initial state $A^+$ can occur
       {\color{blue}{.}}
 \item While the {\color{blue}{CAP-based}} methods
       have the most sound formal basis ...
 \item So far, relativistic decay widths were calculated using
       {\color{blue}{Multichannel}} Multi-Configurational Dirac-Fock
       {\color{blue}{(MMCDF)}} ...
 \item Because of this {\color{blue}{different}} normalization
       the decay widths are not {\color{blue}{amenable to}} ...
 \item We corrected equations (22-24) and the describing text around them.
       They now read:\\
       {\color{blue}{by rewriting the moments (see Eq. (6))
       for the decay width $\Gamma$
       (Eq. (1)) explicitly:
       \begin{align*}
        \tilde{\Gamma}_k &= 2\pi \mu_k \\
               &= 2\pi \sum_j \tilde{E}_j^k
                  \braket{\Phi|\hat{V}|\tilde{\chi}_{\tilde{E}_j}}
                  \braket{\tilde{\chi}_{\tilde{E}_j}|\hat{V}|\Phi} \\           
               &\approx 2\pi \sum_q \left( E^{2h1p}_q\right) ^k
                  \braket{\Phi|\hat{V}|\chi^{2h1p}_q}
                   \braket{\chi^{2h1p}_q|\hat{V}|\Phi}
       \end{align*}
}}
 \item This might be replaced by any {\color{blue}{other}} complete basis ...
 \item {\color{blue}{are}} poles of $\mathbf{\tilde{G}}^-(\omega)$, which can ... 
 \item  ... constructed successively in terms of correlated {\color{blue}{
       $N-1$ particle}} states
 \item ... described reasonably {\color{blue}{well}} in the single-particle picture.
 \item {\color{blue}{The authors}} state that ...
 \item ... results of {\color{blue}{Sukhorukov}} indicate that ...      
 \item We corrected the spelling of German author names in the references.
\end{itemize}

        \closing{Sincerely yours,}
	\end{letter}

\end{document}
