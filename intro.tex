\section{Introduction}
Electronic decay processes are initiated by a sub-outer valence ionization
or excitation. The system relaxes by filling the vacancy with an electron and
transferring the excess energy to another electron of the system, which is emitted.
Popular examples of electronic decay processes are the
well known Auger process \cite{Meitner22,Auger23} and the Interparticle
Coulombic Decay (ICD) process \cite{Cederbaum97,Marburger03}.
They are found in xyz. HIER
In the following, we will assume processes initiated by ionization.

In order to occur, two criteria need to be fulfilled: the energy and the coupling
criterion.
To fulfill the energy criterion the final state energy is required
to be lower than
the energy of the singly ionized initial state. If this is not the case, the
channel defined by a certain doubly ionized final state
is closed and the corresponding fragments of the
channel are not observed after the decay.
To fulfill the coupling criterion, the decay process needs to be fast enough
to prevail over other energetically accessible decay pathways like radiative
relaxation or couping to nuclear degrees of freedom.
It hence contains
the information whether an energetically allowed process can be expected
to be observed experimentally or not.
Therefore, a typical study of electronic decay processes consists of two parts:
\begin{itemize}
 \item determination of the kinetic energy of the secondary electron
       and, as a consequence, which decay channels are open
 \item calculation of the decay width $\Gamma=\frac{\hbar}{\tau}$, which
       is proportional to the decay rate $\frac{1}{\tau}$ and
       inversely proportional to the lifetime $\tau$
\end{itemize}


The primary ionization often removes an electron from an atomic core, where
relativistic effects are stronger than for valence electrons. The relativistic
effects can therefore be expected to play a crucial role for the understanding
of the systems' lifetimes.
Phenomenologically, the relativistic effects can be divided  
into spin-orbit coupling and scalar-relativistic effects. The spin-orbit       
coupling requires the system to be described in terms of the total angular     
momentum $j$ rather than the orbital momentum $l$ and the spin momentum $s$.   
Thereby the non-relativistically degenerate states of one particular $l$       
value are split into two states with $j=l\pm s$ of different energies          
\cite{ReiherWolf09}.                                            
The scalar-relativistic effects result in spatial contractions of all orbitals on
one-electron systems. In many-electron systems, however, those orbitals with density close
to the nucleus are more strongly contracted than others. They thereby
shield the positive charge of the nucleus from the electrons in other orbitals,
which are therefore effectively spatially decontrated compared to the non-relativistic
solutions. Therefore, as a rule of thumb, $s$ and $p$ are spatially contracted while
$d$ and $f$ orbitals are spatially decontrated. \cite{ReiherWolf09}.

The initial and final state energies can be obtained using      
a variety of quantum chemical approaches known in the literature as the       
Algebraic Diagrammatic Construction \cite{Schirmer82_1,Schirmer91,Schirmer98, 
Mertins96_1} (ADC), which is also                               
available for a fully relativistic treatment                    
\cite{Pernpointner04_1,Pernpointner04_2,Pernpointner10_1}.
The calculation of the respective lifetimes is a challenging task, because it
requires the description of both bound and continuum electrons. The bound electrons
are best described by wavefunctions with $\mathcal{L}^2$ boundary conditions,
while continuum electrons far away from the atom are best described by
plane waves. This imposes the technical choice between a) describing the entire
atom using an $\mathcal{L}^2$ basis, b) describing the entire process using
a grid, or c) describing the bound electrons using an $\mathcal{L}^2$ basis
and the continuum electron using a grid.
Either     
of these approaches faces difficulties in either describing the bound or the  
continuum states or some artificially constructed interface region.
Traditionally, most quantum
chemical program packages are based on $\mathcal{L}^2$ bases and we therefore
choose to describe the electronic decay processes using a large $\mathcal{L}^2$
basis for convenience.

The quantum chemical methods, which have been developed for a relativistic
description of the decay widths are the Multichannel Multi-Configurational
Dirac-Fock (MMCDF) \cite{Fritzsche11} and the FanoADC-Stieltjes \cite{Fasshauer15_1}.
The MMCDF is restricted to systems of spherical symmetry and strongly depends on
the manual selection of CI (Configuration
Interaction) components to be included in the description of initial and      
final states. The FanoADC-Stieltjes is able to describe systems of less than
sperical symmetries. Because it is based on the ADC, it includes higher perturbational
orders and is size-consistent. It is therefore a good compromise between
accuracy and computational cost. At the same time, the configurations needed for the
accurate description of initial and final states are determined automatically.
We therefore choose the FanoADC-Stieltjes approach for our calculations.


In this work we focus on the atomic Auger process     
in order                                      
to exploit basic knowledge about the influence of relativistic effects,
but we expect the conclusions to hold for ICD processes as well.
The Auger decay process initiated by a photoionization
can most generally be described by:
\begin{equation*}                                               
 A \quad \xrightarrow{h\nu}\quad (A^+)^* + e^-_{ph} \quad       
    \xrightarrow{Auger} \quad A^{2+} + e^-_{ph} + e^-_{sec}     
\end{equation*}                                                 
                                                                
A system $A$ is ionized, while the photo-electron $e^-_{ph}$ is emitted.      
Afterwards, the actual Auger process of the initial state $A^+$ can occur.    
An electron from an outer                                       
shell fills the vacancy and the excess energy is instantaneously transferred  
to another (secondary) electron $e^-_{sec}$, which              
is subsequently emitted. The final state of the decay process   
is to be described by a doubly charged atom $A^{2+}$ and the secondary        
electron in the continuum.

We have previously shown, how
scalar-relativistic effects influence the decay widths of the Auger process in
noble gas atoms. After primary ionization from the $(n-1)d$ orbitals the nobel
gas atoms decay to $np^{-2}$, $np^{-1}ns^{-1}$ and $ns^{-2}$ final states.
We observed that the Auger decay widths increased by up to \unit[326]{\%}
for radon
by including scalar-relativistic effects in the calculation. This
dramatic increase could be explained by the larger spatial
overlap of the orbitals
involved in the decay compared to non-relativistic calculations due to
a contraction of the $s$ and $p$ orbitals of the final states.\\
However, the fully relativistic calculation resulted in different decay widths
for the different $d_{3/2}^{-1}$ and $d_{5/2}^{-1}$ initial states split by
spin-orbit coupling (see Fig.~5 of Ref. \cite{Fasshauer15_1}).
The aim of this work is therefore to investigate the
influence of spin-orbit coupling on the decay widths of electronic decay processes.
For this purpose, we will study the Auger processes of earth alkaline atoms
after primary ionization from the $(n-1)p$ orbitals. The earth alkaline elements
have the benefit of a single and closed shell $ns^{-2}$ final state.
This allows us to purely observe the effect of the spin-orbit coupling of the
initial state on the decay widths.

The paper is structured as follows:
In section \ref{section:theory} we recapitulate the basics of the
FanoADC-Stieltjes method. We then give the computational details for our
ab initio calculations in section \ref{section:computational}. We present the
results and their interpretation in section \ref{section:results}
and conclude in section \ref{section:conclusions}.
