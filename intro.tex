\section{Introduction}
Electronic decay processes are initiated by a sub-outer valence ionization
or excitation. The system relaxes by filling the vacancy with an electron and
transferring the excess energy to another electron of the system, which is emitted.
Popular examples of electronic decay processes are the
well known Auger process \cite{Meitner22} and the Interparticle
Coulombic Decay (ICD) process \cite{Cederbaum97}.
Depending on the system, they are characterized by lifetimes from attoseconds
to picoseconds. They are therefore often faster than competing decay mechanisms
like radiative relaxation or a coupling to the system's nuclear motion.

The primary ionization often removes an electron from an atomic core, where
relativistic effects are stronger than for valence electrons. The relativistic
effects can therefore be expected to play a crucial role for the understanding
of the systems' lifetimes. We have previously shown, how
scalar-relativistic effects influence the lifetimes of the Auger process in
noble gas atoms. After primary ionization from the $(n-1)d$ orbitals the nobel
gas atoms decay to $np^{-2}$, $np^{-1}ns^{-1}$ and $ns^{-2}$ final states.
We observed that the Auger decay widths increased by up to 600\% for radon
by including scalar-relativistic effects in the calculation. This
dramatic increase could be explained by the larger overlap of the orbitals
involved in the decay compared to non-relativistic calculations due to
a contraction of the $s$ and $p$ orbitals of the final states.\\
However, the fully relativistic calculation resulted in different decay widths
for the different initial states $d_{3/2}^{-1}$ and $d_{5/2}^{-1}$ splitted by
spin-orbit coupling. The aim of this work is therefore to investigate the
influence of spin-orbit coupling on the decay widths of electronic decay processes.
For this purpose, we will study the Auger processes of earth alkaline atoms
after primary ionization from the $(n-1)p$ orbitals. The earth alkaline elements
have the benefit of a single and closed shell $ns^{-2}$ final state.
This allows us to purely study the effect of the spin-orbit coupling of the
initial state on the decay widths.

The calculation of the respective lifetimes is a challenging task, because it
requires the description of both bound and continuum electrons. The bound electrons
are best described by wavefunctions with $\mathcal{L}^2$ boundary conditions,
while continuum electrons far away from the atom are best described by
plane waves. This imposes the technical choice between a) describing the entire
atom using an $\mathcal{L}^2$ basis, b) describing the entire process using
a grid, or c) describing the bound electrons using an $\mathcal{L}^2$ basis
and the continuum electron using a grid. All these options have advantages and
disadvantages that we will not discuss at this point. Traditionally, most quantum
chemical program packages are based on $\mathcal{L}^2$ bases and we therefore
choose to describe the electronic decay processes using a large $\mathcal{L}^2$
basis in this work.

