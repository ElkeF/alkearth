\section{Introduction}
Electronic decay processes are initiated by a sub-outer valence ionization
or excitation. The system relaxes by filling the vacancy with an electron and
transferring the excess energy to another electron of the system, which is emitted.
Popular examples of electronic decay processes are the
well known Auger process \cite{Meitner22,Auger23} and the Interparticle
Coulombic Decay (ICD) process \cite{Cederbaum97,Marburger03}.
Depending on the system, they are characterized by lifetimes from attoseconds
to picoseconds. They are therefore often faster than competing decay mechanisms
like radiative relaxation or a coupling to the system's nuclear motion.

Electronic decay processes in general can occur if
two criteria, the energy and the coupling criterion,
are fulfilled. To fulfill the energy criterion the final state energy is required
to be lower than
the energy of the singly ionized initial state. If this is not the case, the
channel defined by a certain doubly ionized final state
is closed and the corresponding fragments of the
channel are not observed after the decay.
To fulfill the coupling criterion, the decay process needs to be fast enough
to prevail over other energetically accessible decay pathways.
It hence contains
the information whether an energetically allowed process can be expected
to be observed experimentally or not.
Therefore, a typical study of autoionization processes consists of two parts:
\begin{itemize}
 \item determination of the kinetic energy of the secondary electron
       and, as a consequence, which decay channels are open
 \item calculation of the decay width $\Gamma=\frac{\hbar}{\tau}$, which
       is proportional to the decay rate $\frac{1}{\tau}$ and
       inversely proportional to the lifetime $\tau$
\end{itemize}

HIER
In this work we therefore focus on the atomic Auger process     
as a model system in order                                      
to exploit basic knowledge about the influence of relativistic effects.       
The Auger decay process initiated by a photoionization          
can most generally be described by:                             
\begin{equation*}                                               
 A \quad \xrightarrow{h\nu}\quad (A^+)^* + e^-_{ph} \quad       
    \xrightarrow{Auger} \quad A^{2+} + e^-_{ph} + e^-_{sec}     
\end{equation*}                                                 
                                                                
A system $A$ is ionized, while the photo-electron $e^-_{ph}$ is emitted.      
Afterwards, the actual Auger process of the initial state $A^+$ can occur.    
An electron from an outer                                       
shell fills the vacancy and the excess energy is instantaneously transferred  
to another (secondary) electron $e^-_{sec}$, which              
is subsequently emitted. The final state of the decay process   
is to be described by a doubly charged atom $A^{2+}$ and the secondary        
electron in the continuum.

The primary ionization often removes an electron from an atomic core, where
relativistic effects are stronger than for valence electrons. The relativistic
effects can therefore be expected to play a crucial role for the understanding
of the systems' lifetimes.
Phenomenologically, the relativistic effects can be divided  
into spin-orbit coupling and scalar-relativistic effects. The spin-orbit       
coupling requires the system to be described in terms of the total angular     
momentum $j$ rather than the orbital momentum $l$ and the spin momentum $s$.   
Thereby the non-relativistically degenerate states of one particular $l$       
value are split into two states with $j=l\pm s$ of different energies          
\cite{ReiherWolf09}.                                            
The scalar-relativistic effects result in spatial               
contractions of the s and p orbitals                            
and decontraction of d and f orbitals \cite{ReiherWolf09}.

The initial and final state energies can be obtained using      
a variety of quantum chemical approaches known in the literature as the       
Algebraic Diagrammatic Construction \cite{Schirmer82_1,Schirmer91,Schirmer98, 
Mertins96_1} (ADC), which is also                               
available for a fully relativistic treatment                    
\cite{Pernpointner04_1,Pernpointner04_2,Pernpointner10_1}.
The calculation of the respective lifetimes is a challenging task, because it
requires the description of both bound and continuum electrons. The bound electrons
are best described by wavefunctions with $\mathcal{L}^2$ boundary conditions,
while continuum electrons far away from the atom are best described by
plane waves. This imposes the technical choice between a) describing the entire
atom using an $\mathcal{L}^2$ basis, b) describing the entire process using
a grid, or c) describing the bound electrons using an $\mathcal{L}^2$ basis
and the continuum electron using a grid.
Either     
of these approaches faces difficulties in either describing the bound or the  
continuum states or some artificially constructed interface region.
Traditionally, most quantum
chemical program packages are based on $\mathcal{L}^2$ bases and we therefore
choose to describe the electronic decay processes using a large $\mathcal{L}^2$
basis for convenience.

The quantum chemical methods, which have been used for non-relativistic       
descriptions of the decay widths are the Wigner-Weisskopf       
theory \cite{Santra02}, CAP-CI                                  
(Complex Absorbing Potential based on a Configuration Interaction wavefunction)
\cite{SakuraiModern94,Santra01_3}, CAP-ADC \cite{Vaval07},      
CAP/EOM-CCSD (Equation of Motion Coupled Cluster with Singles and Doubles)    
\cite{Ghosh14} and the FanoADC-Stieltjes method                 
\cite{Averbukh05}.                                              
While the CAP-based methods have the most sound formal basis    
of the methods above, they suffer from wrong densities and populations for    
many-particle systems                                           
and are computationally expensive at the same time.             
In contrast to this, the                                        
Wigner-Weisskopf theory is based on the lowest                  
non-vanishing order of perturbation                             
theory and therefore computationally affordable even for large systems.       
However, the price for the lower                                
computational costs are less accurate results. A compromise between
accuracy and computational cost is the FanoADC-Stieltjes approach, which      
is based on the {ADC} and therefore includes                    
higher perturbational orders and is size-consistent.            
For these reasons,                                              
the FanoADC-Stieltjes method was implemented in the relativistic
quantum chemical program package Dirac \cite{DIRAC13}. We present first results
obtained with the relativistic FanoADC-Stieltjes method in this work.

HIER
So far, relativistic decay widths were calculated using         
Multichannel Multi-Configurational Dirac-Fock (MMCDF) \cite{Fritzsche11}.     
However, this approach                                          
does not only highly depend on manual selection of CI (Configuration
Interaction) components to be included in the description of initial and      
final state, it moreover is not capable of describing systems with less       
than spherical symmetry. This reduces the applicability of the method to      
atomic Auger processes, but allows us to compare our results obtained with    
the relativistic FanoADC-Stieltjes method for exactly this special case.

We have previously shown, how
scalar-relativistic effects influence the lifetimes of the Auger process in
noble gas atoms. After primary ionization from the $(n-1)d$ orbitals the nobel
gas atoms decay to $np^{-2}$, $np^{-1}ns^{-1}$ and $ns^{-2}$ final states.
We observed that the Auger decay widths increased by up to \unit[326]{\%}
for radon
by including scalar-relativistic effects in the calculation. This
dramatic increase could be explained by the larger overlap of the orbitals
involved in the decay compared to non-relativistic calculations due to
a contraction of the $s$ and $p$ orbitals of the final states.\\
However, the fully relativistic calculation resulted in different decay widths
for the different initial states $d_{3/2}^{-1}$ and $d_{5/2}^{-1}$ splitted by
spin-orbit coupling. The aim of this work is therefore to investigate the
influence of spin-orbit coupling on the decay widths of electronic decay processes.
For this purpose, we will study the Auger processes of earth alkaline atoms
after primary ionization from the $(n-1)p$ orbitals. The earth alkaline elements
have the benefit of a single and closed shell $ns^{-2}$ final state.
This allows us to purely study the effect of the spin-orbit coupling of the
initial state on the decay widths.

The paper is structured as follows:
In section \ref{section:theory} we recapitulate the basics of the
FanoADC-Stieltjes method. We then give the computational details for our
ab initio calculations in section \ref{section:computational}. We present the
results and their interpretation in section \ref{section:results}
and conclude in section \ref{section:conclusions}.
