\section{Conclusions}
\label{section:conclusions}

We have presented lower bounds for
decay widths for the Auger-Meitner process initiated by
ionization from the $(n-1)p$ orbitals of strontium and radium.
Through analysis of results from different Hamiltonians and initial state
eigenvectors as well as radial densities of orbitals involved in the
Auger-Meitner process, we were able to show the importance of configuration interaction
in this specific case and the effect of spin-orbit coupling on decay widths
of electronic decay processes in general.
We condensed our findings into the following rule of thumb for the decay widths
of electronic decay processes:
Two ionized initial states that stem from the same non-relativistic configuration and
are split by spin-orbit coupling will have different decay widths, where the decay width
of the $l-\frac12$ initial state will be significantly lower than the decay width of
the $l + \frac12$ initial state.
\textcolor{blue}
{We have tested this rule of thumb against Auger-Meitner decay widths available in the
literature and could thereby validate it for the majority
of cases.}
