\section{Computational Details}
\label{section:computational}
The auger decay widths were calculated with the relativistic FanoADC-Stieltjes
method
implemented in the relativistic quantum chemical program DIRAC \cite{DIRAC17}.
We included up to third order contributions of perturbation theory and additional
constant diagrams.
For each element four-component calculations based on the
Dirac-Coulomb (DC)Hamiltonian
and scalarrelativistic spinfree calculations were
performed for both the $(n-1)p_{1/2}$ and $(n-1)p_{3/2}$ initial states.
Dyall's cv4z basis sets \cite{Dyall4s-7s09} were augmented with additional diffuse
5s5p5d3f
basis functions following the Kaufmann-Baumeister-Jungen approach
\cite{Kaufmann89}.
The resulting moments were checked for numerical instabilities.
Only those moments, without numerical instabilities entered the interpolation
scheme for the determination of the decay widths.\\
The radial orbital densities of the ions were calculated using GRASP
\cite{Parpia96,xyz}.
