\section{Results}
\label{section:results}

\begin{figure}[h]
 \centering
 \includegraphics[width=0.9\columnwidth]{pics/Sr_rel_sdip.pdf}
 \caption{Comparison of the single (SIP) and double (DIP) ionization spectra
          of the strontium obtained by a DC-ADC calculation.}
 \label{fig:sdip}
\end{figure}

\begin{table}[htb]
 \centering
 \caption{Ionization energies, pole-strengths (ps) and decay widths $\Gamma$ of
          different Auger-Meitner initial states of strontium and radium obtained
          by relativistic FanoADC-Stieltjes calculations.}
 \begin{tabular}{lrrr}
  \toprule
   initial state    & energy $[\unit{eV}]$ & ps & $\Gamma [\unit{meV}]$\\
  \midrule
   Sr spinfree      & 28.599 & 0.78 &   0.56\\  
   Sr$4p_{1/2,1/2}$ & 29.402 & 0.80 &   0.10\\
   Sr$4p_{3/2,1/2}$ & 28.277 & 0.76 &   1.23\\
   Sr$4p_{3/2,3/2}$ & 28.277 & 0.76 &   1.17\\
%  \midrule
%   Ba$5p_{1/2,1/2}$ & 25.108 & 0.80 & unreliable\\
%   Ba$5p_{3/2,1/2}$ & 23.106 & 0.76 &   55.9\\
%   Ba$5p_{3/2,3/2}$ & 23.106 & 0.76 &   63.1\\
  \midrule
   Ra spinfree      & 21.836 & 0.49 &  28.56 \\  
   Ra$6p_{1/2,1/2}$ & 25.494 & 0.78 &   0.26\\
   Ra$6p_{3/2,1/2}$ & 19.267 & 0.50 &  93.16 \\
   Ra$6p_{3/2,3/2}$ & 19.267 & 0.50 &  98.86\\
  \bottomrule
 \end{tabular}
 \label{tab:widths}
\end{table}



\begin{align*}
 (n-1)p^{-1} \,ns^2         \rightarrow & (n-1)p^6 + e^- \\
 (n-1)p^{-1} \,(n-1)d \, ns \rightarrow & (n-1)p^6 + e^- \\
 (n-1)p^{-1} \,(n-1)d^2     \rightarrow & (n-1)p^6 + e^- \\
\end{align*}


\begin{figure}[h]
 \centering
 \includegraphics[width=\columnwidth]{pics/sr_ba_ra.pdf}
\end{figure}

\begin{figure}[h]
 \centering
 \includegraphics[width=\columnwidth]{pics/sr_ion_R.pdf}\\
 %\includegraphics[width=\columnwidth]{pics/ba_ion_R.pdf}\\
 \includegraphics[width=\columnwidth]{pics/ra_ion_R.pdf}\\
 \caption{Radial densities of the orbitals of the $(n-1)p^5s^2$ ions
          involved in the Auger decay.
          The expectation value of the electrons position of the $(n-1)p_{1/2}$
          orbital is lower than of the respective $(n-1)p_{3/2}$
          orbitals. The $ns$ orbitals of the ions experience a stronger
          contraction the those of the atom (not shown here).}
 \label{fig:radial}
\end{figure}




\begin{figure}[h]
 \centering
 \includegraphics[width=\columnwidth]{pics/ra_6d_R.pdf}
 \caption{Radial densities of the radium orbitals of the ionic $6p^5 6d 7s$
          configuration involved in the Auger decay. The overlap of the radial
          densities of the $6p$ orbitals with the radial densities of the $6d$ orbitals
          is more pronounced than with the radial density of the $7s$ orbital.}
 \label{fig:radial}
\end{figure}
