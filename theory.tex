\section{Theory}
\label{section:theory}

Following Wentzel \cite{Wentzel27} and later Feshbach \cite{Feshbach58,Feshbach62}
and Fano \cite{Fano61}
the decay width of a decay process initiated by a
primary ionization is given by 

\begin{equation} \label{equation:Fano_golden}
  \Gamma = \sum_\beta 2\pi
           \left| \braket{\Phi|\hat{V}|\chi_{\beta,\varepsilon}} \right|^2 .
\end{equation}

Here, $\ket{\Phi}$ and $\ket{\chi_{\beta,\varepsilon}}$ denote the initial and
final state, respectively. $\hat{V}$ is the interaction operator of the
initial and final states, which in Feshbach's definition is known as $H_{PQ}$.
The index $\beta$ refers to the different
decay channels and $\varepsilon$ denotes the energy of the final state.
Eq. (\ref{equation:Fano_golden}) thereby connects the metastable initial
and the continuum final states. They are constructed by partitioning the
Hamiltonian into two subspaces. The initial (final) state is then an
eigenfunction of this initial (final) state sub-space Hamiltonian.
However, finding proper solutions to both the initial and the final
states on an equal footing is a non-trivial task, because they adhere to
different boundary conditions. Since the final state depends on the energy
of the emitted electron, any approach needs to either determine the continuum
state or to mimic the final state using $\mathcal{L}^2$-functions.
While the continuum functions are normalized with respect to their energy

\begin{equation}
 \braket{\chi_\varepsilon| \chi_{\varepsilon'}} = \delta(\varepsilon-\varepsilon')
\end{equation}

the $\mathcal{L}^2$ approach is based on a discrete set of final states
$\ket{\tilde{\chi}_{\tilde{E}}}$
which adhere to different boundary
conditions and are normalized with respect to space (see e.g. \cite{Craigie14})
\begin{equation}
 \braket{ \tilde{\chi}_{\tilde{E}_i} | \tilde{\chi}_{\tilde{E}_j} } = \delta_{ij}.
\end{equation}

Because of this different normalization the decay widths are not amenable to
a direct calculation. As first proposed by Hazi \cite{hazi1978}, for
autoionization processes such difficulties
can be solved by using the
Stieltjes-Chebyshev moment theory also called Stieltjes imaging
\cite{Langhoff76,Corcoran77,MuellerPlathe90}.
It relies on the observation that the moments of order $k$ of the projected final
state Hamiltonian $H_f$

\begin{equation}
 \mu_k = \braket{ \Phi | \hat{V} H^k_f \hat{V} | \Phi }
\end{equation}

calculated from the determined
discrete pseudo-spectrum are good approximations to the moments determined
from the real continuum states.
This can be shown by inserting the resolution of identity for
the continuum states

\begin{equation}
 \mu_k = \sum_i \varepsilon_i^k
         \left| \braket{ \Phi | \hat{V} | \chi_{i,\varepsilon} } \right| ^2
       + \int\limits_{E_{0}}^{\infty} \varepsilon^k
         \left| \braket{\Phi|\hat{V}|\chi_{\varepsilon}} \right|^2 \mathrm{d}\varepsilon  .
\end{equation}

Since the non-zero contribution to the coupling matrix elements in the
Feshbach-Fano approach stems only
from an interaction region of finite size, where the $\mathcal{L}^2$ final
state functions are nonvanishing, we may replace the expansion
$\sum\limits_i \ket{\chi_{i,\varepsilon}} \bra{\chi_{i,\varepsilon}}
 + \int \mathrm{d}\varepsilon \ket{\chi_\varepsilon} \bra{\chi_\varepsilon}$
by its $\mathcal{L}^2$ approximation
$\sum\limits_j \ket{\tilde{\chi}_{\tilde{E}_j}} \bra{\tilde{\chi}_{\tilde{E}_j}}$
(see \cite{Reinhardt79})

\begin{equation}
 \label{eq:moment_discrete}
 \mu_k \approx \sum\limits_j \tilde{E}_j ^k
         \left| \braket{\Phi|\hat{V}|\tilde{\chi}_{\tilde{E}_j}}  \right|^2 .
\end{equation}

Then the decay width can be determined through a series of
consecutive approximations to the moments of increasing order $k$.

To achieve this kind of description, we choose the relativistic
FanoADC-Stieltjes approach.
Here, a proper selection of $2h1p$ intermediate state configurations are used
for the description of the continuum final state,
while the rest is used for the description
of the initial state.
The resulting discrete pseudo-spectrum
is then subject to a Stieltjes imaging procedure.
An exhaustive description of the method can be found in Refs. \cite{Fasshauer15_1}
and \cite{Fasshauer_thesis}.
